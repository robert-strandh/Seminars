\documentclass{beamer}
\usepackage[latin1]{inputenc}
\beamertemplateshadingbackground{red!10}{blue!10}
%\usepackage{fancybox}
\usepackage{epsfig}
\usepackage{verbatim}
\usepackage{url}
%\usepackage{graphics}
%\usepackage{xcolor}
\usepackage{fancybox}
\usepackage{moreverb}
%\usepackage[all]{xy}
\usepackage{listings}
\usepackage{filecontents}
\usepackage{graphicx}

\lstset{
  language=Lisp,
  basicstyle=\scriptsize\ttfamily,
  keywordstyle={},
  commentstyle={},
  stringstyle={}}

\def\inputfig#1{\input #1}
\def\inputeps#1{\includegraphics{#1}}
\def\inputtex#1{\input #1}

\inputtex{logos.tex}

%\definecolor{ORANGE}{named}{Orange}

\definecolor{GREEN}{rgb}{0,0.8,0}
\definecolor{YELLOW}{rgb}{1,1,0}
\definecolor{ORANGE}{rgb}{1,0.647,0}
\definecolor{PURPLE}{rgb}{0.627,0.126,0.941}
\definecolor{PURPLE}{named}{purple}
\definecolor{PINK}{rgb}{1,0.412,0.706}
\definecolor{WHEAT}{rgb}{1,0.8,0.6}
\definecolor{BLUE}{rgb}{0,0,1}
\definecolor{GRAY}{named}{gray}
\definecolor{CYAN}{named}{cyan}

\newcommand{\orchid}[1]{\textcolor{Orchid}{#1}}
\newcommand{\defun}[1]{\orchid{#1}}

\newcommand{\BROWN}[1]{\textcolor{BROWN}{#1}}
\newcommand{\RED}[1]{\textcolor{red}{#1}}
\newcommand{\YELLOW}[1]{\textcolor{YELLOW}{#1}}
\newcommand{\PINK}[1]{\textcolor{PINK}{#1}}
\newcommand{\WHEAT}[1]{\textcolor{wheat}{#1}}
\newcommand{\GREEN}[1]{\textcolor{GREEN}{#1}}
\newcommand{\PURPLE}[1]{\textcolor{PURPLE}{#1}}
\newcommand{\BLACK}[1]{\textcolor{black}{#1}}
\newcommand{\WHITE}[1]{\textcolor{WHITE}{#1}}
\newcommand{\MAGENTA}[1]{\textcolor{MAGENTA}{#1}}
\newcommand{\ORANGE}[1]{\textcolor{ORANGE}{#1}}
\newcommand{\BLUE}[1]{\textcolor{BLUE}{#1}}
\newcommand{\GRAY}[1]{\textcolor{gray}{#1}}
\newcommand{\CYAN}[1]{\textcolor{cyan }{#1}}

\newcommand{\reference}[2]{\textcolor{PINK}{[#1~#2]}}
%\newcommand{\vect}[1]{\stackrel{\rightarrow}{#1}}

% Use some nice templates
\beamertemplatetransparentcovereddynamic

\newcommand{\A}{{\mathbb A}}
\newcommand{\degr}{\mathrm{deg}}

\title{Choosing a programming language}

\author{Robert Strandh}
\institute{
University of Bordeaux
}
\date{May, 2018}

%\inputtex{macros.tex}


\begin{document}
\frame{
\resizebox{3cm}{!}{\includegraphics{Logobx.pdf}}
\hfill
\resizebox{1.5cm}{!}{\includegraphics{labri-logo.pdf}}
\titlepage
\vfill
\small{Dfind, G�teborg}
}

\setbeamertemplate{footline}{
\vspace{-1em}
\hspace*{1ex}{~} \GRAY{\insertframenumber/\inserttotalframenumber}
}

\begin{frame}
\frametitle{Overview of talk}
\begin{itemize}
\item Programming language characteristics.
\item Common misconceptions.
\item Requirements for making a good choice.
\item Risk analysis.
\end{itemize}
\end{frame}

%-----------------------------------------------------------
\begin{frame}\frametitle{How the choice is often made}

The choice of programming language for a project (when there are
several possibilities) is often based on \emph{gut feeling}.
\vskip 0.5cm
Often, no real analysis is made, because the decision maker:

\begin{itemize}
\item has only partial knowledge of the characteristics of possible
  choices;
\item sometimes has incorrect information about the possible choices;
\item has insufficient training to appreciate the characteristics of
  possible choices;
\end{itemize}

\end{frame}
%-----------------------------------------------------------
\begin{frame}\frametitle{How the choice is often made}

Often, no real analysis is made, because the decision maker:

\begin{itemize}
\item has insufficient experience with the possible choices to
  determine which choice is adapted to the current project;
\item has insufficient information about the cost associated with
  training staff in a new language vs the productivity advantages of
  that language;
\item has insufficient information about the cost associated with
  hiring new staff for a new language vs the productivity advantages of
  that language.
\end{itemize}

\end{frame}
%-----------------------------------------------------------
\begin{frame}\frametitle{Language vs implementation}

\end{frame}
%-----------------------------------------------------------
\begin{frame}\frametitle{Language vs implementation}

Language: A description of the syntax and semantics of conforming
programs, and of consequences of using non-conforming constructs.
\vskip 0.5cm
Example of the latter: In C, obtaining a pointer outside of an array
has undefined consequences.  (It is interesting to contemplate why.)

\end{frame}
%-----------------------------------------------------------
\begin{frame}\frametitle{Language vs implementation}

Implementation: Software that accepts conforming programs and executes
them according to the language semantics, and that reports
non-conforming constructs where required by the language definition.

\end{frame}
%-----------------------------------------------------------
\begin{frame}\frametitle{Language vs implementation}

In many cases, the language definition does not require the compiler
to check for non-conforming constructs.  Why?
\vskip 0.5cm
As a consequence, many non-conforming programs go undetected.

\end{frame}
%-----------------------------------------------------------
\begin{frame}\frametitle{Language vs implementation}

The distinction language/implementation is sometimes blurred:

\begin{itemize}
\item Some languages are defined by a \emph{reference
  implementation}.  Examples?
\item Some language definitions are controlled by the same
  organization that supplies some dominating implementation.
  Examples?
\end{itemize}

\end{frame}
%-----------------------------------------------------------
\begin{frame}\frametitle{Language vs implementation}

The distinction language/implementation is sometimes blurred:

\begin{itemize}
\item Some languages are defined by a \emph{reference
  implementation}.  Perl 6,
\item Some language definitions are controlled by the same
  organization that supplies some dominating implementation.
  Examples?
\end{itemize}

\end{frame}
%-----------------------------------------------------------
\begin{frame}\frametitle{Language characteristics}

\end{frame}
%-----------------------------------------------------------
\begin{frame}\frametitle{Strong vs weak typing}

A language can be \emph{strongly typed} or \emph{weakly typed}, and
even \emph{untyped}.
\vskip 0.5cm
Strongly typed: It is impossible for an object of one type to be
mistaken for an object of a different type.  Either the compiler made
sure no mistake is possible, or the run-time system checked it.
Examples: Java, C\#, Common Lisp.
\vskip 0.5cm
Weakly typed: No such guarantees are made.
Example: C
\vskip 0.5cm
Untyped: Data can be interpreted differently by different operations.
Example: Assembler

\end{frame}
%-----------------------------------------------------------
\begin{frame}\frametitle{Static vs dynamic typing}

A language can be \emph{statically typed} or \emph{dynamically
  typed}.
\vskip 0.5cm
Static typing: Type information is associated with the
\emph{variables} in the program.  Type checking or type inference is
always handled at compile time.
\vskip 0.5cm
Dynamic typing: Type information is associated with the \emph{objects}
manipulated by the program.  Type checking must sometimes be done at
execution time.  Not always?

\end{frame}
%-----------------------------------------------------------
\begin{frame}\frametitle{Manifest vs implicit typing}

A statically typed programming language may be based on \emph{manifest
  typing} or \emph{implicit typing}.
\vskip 0.5cm
Manifest typing: The programmers supplies the type of the variables
explicitly.
\vskip 0.5cm
Implicit typing (sometimes called \emph{latent} typing): The compiler
\emph{infers} the type of the variables from the operations it
participates in.

\end{frame}
%-----------------------------------------------------------
\begin{frame}\frametitle{Manifest vs implicit typing}

Why is the distinction between manifest and implicit typing important?
\vskip 0.5cm
With manifest typing, the programmer is given too much responsibility
too early in the development process.  Choices made may easily change
later on.

\end{frame}
%-----------------------------------------------------------
\begin{frame}\frametitle{Manifest vs implicit typing}

Some dynamically typed programming languages allow optional type
declarations.
\vskip 0.5cm
Such declarations are sometimes used in order to allow the compiler to
generate faster code.

\end{frame}
%-----------------------------------------------------------
\begin{frame}\frametitle{Static vs dynamic}

A programming language can be either \emph{static} or \emph{dynamic}.
\vskip 0.5cm
Static: There is a clear distinction between \emph{compile time} and
\emph{run time}.  Code is \emph{generated} at compile time and
\emph{executed} at run time.
\vskip 0.5cm
Dynamic: There is no clear distinction between \emph{compile time} and
\emph{run time}.  The program might change as a result of executing
code (for example, executing a statement that defines a function or a
class).

\end{frame}
%-----------------------------------------------------------
\begin{frame}\frametitle{Interpreted vs compiled}

Can you guess what an \emph{interpreted programming language} is and
what a \emph{compiled programming language} might be?

\end{frame}
%-----------------------------------------------------------
\begin{frame}\frametitle{Interpreted vs compiled}

Can you guess what an \emph{interpreted programming language} is and
what a \emph{compiled programming language} might be?
\vskip 0.5cm
A programming \emph{language} is neither interpreted nor compiled.  A
programming language \emph{implementation} is either one or both (it
is a spectrum of possibilities).

\end{frame}
%-----------------------------------------------------------
\begin{frame}\frametitle{Interpreted vs compiled}

The distinction is important because it affects the \emph{execution
  performance} of the program.
\vskip 0.5cm
An implementation that compiles to native code has the potential of
generating \emph{fast} code.
\vskip 0.5cm
An implementation that has more elements of interpretation can
generate code that is slower by a factor 10 or more compared to code
generated by a native compiler.

\end{frame}
%-----------------------------------------------------------
\begin{frame}\frametitle{Manual vs automatic memory management}

Manual memory management: The language requires the programmer to
de-allocate objects that are no longer going to be referenced.
Examples: C, C++
\vskip 0.5cm
Automatic memory management: The implementation of the language is
required to automatically recycle objects that are no longer
referenced.  Examples: Java, C\#, Common Lisp, Haskell, OCaml,
JavaScript.

\end{frame}
%-----------------------------------------------------------
\begin{frame}\frametitle{Standardization}

A language is said to have an \emph{independent standard} if and only
if the definition of the language is published by an organization
other than a supplier of an implementation.

\end{frame}
%-----------------------------------------------------------
\begin{frame}\frametitle{Scripting languages}

There are few properties that characterize scripting languages.
Probably only:

\begin{itemize}
\item The creators meant for the language to be used for scripting.
\item It is a dynamic language.
\end{itemize}

Common scripting languages are single-implementation languages with
the implementation written as a slow interpreter.  This technique is
considered acceptable because of the first item above.

\end{frame}
%-----------------------------------------------------------
\begin{frame}\frametitle{Scripting languages}

Typically, a static language is used for the main body of code, and a
``scripting language'' for, um, scripting.
\vskip 0.5cm
When the advanced user starts writing serious code using the scripting
language (because that's the only choice possible), the combined
result is slow despite the best intentions of the creators.
\vskip 0.5cm
Furthermore, debugging code written in two languages is typically hard.

\end{frame}
%-----------------------------------------------------------
\begin{frame}\frametitle{Domain-specific languages}

A domain-specific language is a language that was designed for a
particular family of programming tasks.
\vskip 0.5cm
Often, the language is defined by the same organization that then uses
it.
\vskip 0.5cm
The productivity advantage can be huge.
\vskip 0.5cm
Designing and implementing a domain-specific language requires
expertise in language design and compiler technology.

\end{frame}
%-----------------------------------------------------------
\begin{frame}\frametitle{Common misconceptions}

\end{frame}
%-----------------------------------------------------------
\begin{frame}\frametitle{Manual vs automatic memory management}

Common misconception: manual memory management is faster than
automatic memory management.
\vskip 0.5cm
With manual memory management, for modularity, it is often necessary
to copy objects or to use reference counters.
\vskip 0.5cm
Such techniques can easily incur a performance penalty of a factor
10--100 on modern hardware.  Why?
\vskip 0.5cm
Modern garbage collectors are fast, concurrent, multi-threaded, and
some have real-time or near-real-time guarantees.

\end{frame}
%-----------------------------------------------------------
\begin{frame}\frametitle{Compiled/static vs interpreted/dynamic}

Common misconception: dynamic languages must be interpreted.
\vskip 0.5cm
Conversely: only static languages can be compiled.
\vskip 0.5cm
Modern implementations of dynamic languages generate native code
\emph{on the fly}.

\end{frame}
%-----------------------------------------------------------
\begin{frame}\frametitle{Alternatives to scripting languages}

Use the same dynamic language for the main body of code and for
scripting purposes.
\vskip 0.5cm
Choose a language that has an efficient implementation that compiles
to native code.

\end{frame}
%-----------------------------------------------------------
\begin{frame}\frametitle{So how do we choose a language?}

\end{frame}
%-----------------------------------------------------------
\begin{frame}\frametitle{So how do we choose a language?}

Making a good language choice requires:

\begin{itemize}
\item Good knowledge of the characteristics of several programming
  languages (the subject of this talk).
\item Good knowledge of the nature of the task to be accomplished.
\item A separate and detailed budget for each ``reasonable'' language
  choice.
\end{itemize}

We will look at the last item a bit more.

\end{frame}
%-----------------------------------------------------------
\begin{frame}\frametitle{How not to choose}

  \begin{itemize}
  \item ``We need all the speed we can get, and it is known that the
    C++ compiler generates very fast code.  Therefore we choose C++.''
  \item ``All our programmers already know Java.  Therefore we choose
    Java.''
  \item ``We have made a huge investment in programming tools for
    C\#.  Therefore we choose C\#.''
  \end{itemize}

\end{frame}
%-----------------------------------------------------------
\begin{frame}\frametitle{What to include in the budgets}

  \begin{itemize}
  \item Cost of acquisition of language tools.
  \item Estimated development and maintenance cost.
  \item Cost of training and hiring new staff.
  \item ...
  \item Risk analysis.
  \end{itemize}

\end{frame}
%-----------------------------------------------------------
\begin{frame}\frametitle{Risk analysis}

For every major possible choice (tools, staff, development method,
etc.), make a list of possible events that might have a negative
impact on the project.
\vskip 0.5cm
For each event, state:

\begin{itemize}
\item its likelihood,
\item the cost to the project if nothing is done,
\item actions to avoid the negative impact, and
\item the cost of those actions.
\end{itemize}

\end{frame}
%-----------------------------------------------------------
\begin{frame}\frametitle{Risk analysis}

Example:
\vskip 0.5cm
Choice: Make Joe a member of the staff.  Joe is a reckless driver.
\vskip 0.5cm
Event: Joe has a traffic accident and can no longer work on the
project.
\vskip 0.5cm
Likelihood: Unlikely
\vskip 0.5cm
Cost if nothing is done: The project will be delayed by six months.
\vskip 0.5cm
Action: Hire a replacement for Joe.
\vskip 0.5cm
Cost of action: Salary, training, etc.

\end{frame}
%-----------------------------------------------------------
\begin{frame}\frametitle{Risk analysis}

Example:
\vskip 0.5cm
Choice: Use the language C\#.
\vskip 0.5cm
Event: Microsoft is bought by Apple (or Google) and C\# is no longer
supported.
\vskip 0.5cm
Likelihood: Unlikely
\vskip 0.5cm
Cost if nothing is done: All code must be rewritten in Java.
\vskip 0.5cm
Action: Obtain (buy, develop) a replacement for Microsoft C\#.
\vskip 0.5cm
Cost of action: Cost of purchase or development.

\end{frame}
%-----------------------------------------------------------
\begin{frame}\frametitle{Standardization and risk analysis}

If a language does not have an independent standard, its specification
may change as a result of the organization that supplies the
implementation.
\vskip 0.5cm
The cost to a project could be huge.  Much code may need to be
rewritten, perhaps not immediately, but over time.
\vskip 0.5cm
Such possibilities must be taken into account in the risk analysis.
\vskip 0.5cm
Examples?

\end{frame}
%-----------------------------------------------------------
\begin{frame}\frametitle{Availability of good implementations}

A good implementation may exist when a project is started, but might
then be abandoned over time.
\vskip 0.5cm
Again, the cost to a project could be huge, including a complete
rewrite using a different language.
\vskip 0.5cm
Such possibilities must be taken into account in the risk analysis.
\vskip 0.5cm
Examples?

\end{frame}
%-----------------------------------------------------------
\begin{frame}\frametitle{Availability of programmers}

New graduates may only know a few languages.
\vskip 0.5cm
They may also have the wrong idea of languages they do not (yet) know
and sometimes also about the languages they (think they) know.
\vskip 0.5cm
For other languages, it may be necessary to train existing programmers
or to hire new programmers.
\vskip 0.5cm
The salary may need to be higher.  That will be another factor in the
cost analysis of the project.

\end{frame}
%-----------------------------------------------------------
\begin{frame}\frametitle{Availability of programmers}

For a medium-sized or large company, it is advisable to have
programmers with knowledge of several different programming languages
using different programming paradigms.
\vskip 0.5cm
That way, a large spectrum of languages can be covered.
\vskip 0.5cm
Programmers can participate in decisions about programming languages,
and they can help train other programmers.

\end{frame}
%-----------------------------------------------------------
\begin{frame}\frametitle{Conclusions}

Choosing the right language for a task can have a great impact on the
amount of work it takes to finish that task.
\vskip 0.5cm
Making the right choice requires expertise that must either be
developed in-house, or hired when a new project is starting.
\vskip 0.5cm
It is best to do a detailed cost/benefit analysis, including a risk
analysis, for each reasonable choice of a language.

\end{frame}
%-----------------------------------------------------------
\begin{frame}\frametitle{Questions?}

Thank you for listening!
\vskip 0.5cm
Do you have any questions?

\end{frame}
%-----------------------------------------------------------
\end{document}
