\documentclass{beamer}
\usepackage[latin1]{inputenc}
\beamertemplateshadingbackground{red!10}{blue!10}
%\usepackage{fancybox}
\usepackage{epsfig}
\usepackage{verbatim}
\usepackage{url}
%\usepackage{graphics}
%\usepackage{xcolor}
\usepackage{fancybox}
\usepackage{moreverb}
%\usepackage[all]{xy}
\usepackage{listings}
\usepackage{filecontents}
\usepackage{graphicx}

\lstset{
  language=Lisp,
  basicstyle=\scriptsize\ttfamily,
  keywordstyle={},
  commentstyle={},
  stringstyle={}}

\def\inputfig#1{\input #1}
\def\inputeps#1{\includegraphics{#1}}
\def\inputtex#1{\input #1}

\inputtex{logos.tex}

%\definecolor{ORANGE}{named}{Orange}

\definecolor{GREEN}{rgb}{0,0.8,0}
\definecolor{YELLOW}{rgb}{1,1,0}
\definecolor{ORANGE}{rgb}{1,0.647,0}
\definecolor{PURPLE}{rgb}{0.627,0.126,0.941}
\definecolor{PURPLE}{named}{purple}
\definecolor{PINK}{rgb}{1,0.412,0.706}
\definecolor{WHEAT}{rgb}{1,0.8,0.6}
\definecolor{BLUE}{rgb}{0,0,1}
\definecolor{GRAY}{named}{gray}
\definecolor{CYAN}{named}{cyan}

\newcommand{\orchid}[1]{\textcolor{Orchid}{#1}}
\newcommand{\defun}[1]{\orchid{#1}}

\newcommand{\BROWN}[1]{\textcolor{BROWN}{#1}}
\newcommand{\RED}[1]{\textcolor{red}{#1}}
\newcommand{\YELLOW}[1]{\textcolor{YELLOW}{#1}}
\newcommand{\PINK}[1]{\textcolor{PINK}{#1}}
\newcommand{\WHEAT}[1]{\textcolor{wheat}{#1}}
\newcommand{\GREEN}[1]{\textcolor{GREEN}{#1}}
\newcommand{\PURPLE}[1]{\textcolor{PURPLE}{#1}}
\newcommand{\BLACK}[1]{\textcolor{black}{#1}}
\newcommand{\WHITE}[1]{\textcolor{WHITE}{#1}}
\newcommand{\MAGENTA}[1]{\textcolor{MAGENTA}{#1}}
\newcommand{\ORANGE}[1]{\textcolor{ORANGE}{#1}}
\newcommand{\BLUE}[1]{\textcolor{BLUE}{#1}}
\newcommand{\GRAY}[1]{\textcolor{gray}{#1}}
\newcommand{\CYAN}[1]{\textcolor{cyan }{#1}}

\newcommand{\reference}[2]{\textcolor{PINK}{[#1~#2]}}
%\newcommand{\vect}[1]{\stackrel{\rightarrow}{#1}}

% Use some nice templates
\beamertemplatetransparentcovereddynamic

\newcommand{\A}{{\mathbb A}}
\newcommand{\degr}{\mathrm{deg}}

\title{Choosing a programming language}

\author{Robert Strandh}
\institute{
University of Bordeaux
}
\date{May, 2018}

%\inputtex{macros.tex}


\begin{document}
\frame{
\resizebox{3cm}{!}{\includegraphics{Logobx.pdf}}
\hfill
\resizebox{1.5cm}{!}{\includegraphics{labri-logo.pdf}}
\titlepage
\vfill
\small{Dfind, G�teborg}
}

\setbeamertemplate{footline}{
\vspace{-1em}
\hspace*{1ex}{~} \GRAY{\insertframenumber/\inserttotalframenumber}
}

\begin{frame}
\frametitle{Overview of talk}
\begin{itemize}
\item Programming language characteristics.
\item Common misconceptions.
\item Requirements for making a good choice.
\end{itemize}
\end{frame}

%-----------------------------------------------------------
\begin{frame}\frametitle{How the choice is often made}

The choice of programming language for a project (when there are
several possibilities) is often based on \emph{gut feeling}.

Often, no real analysis is made, because the decision maker:

\begin{itemize}
\item has only partial knowledge of the characteristics of possible
  choices;
\item sometimes has incorrect information about the possible choices;
\item has insufficient training to appreciate the characteristics of
  possible choices;
\end{itemize}

\end{frame}
%-----------------------------------------------------------
\begin{frame}\frametitle{How the choice is often made}

Often, no real analysis is made, because the decision maker:

\begin{itemize}
\item has insufficient experience with the possible choices to
  determine which choice is adapted to the current project;
\item has insufficient information about the cost associated with
  training staff in a new language vs the productivity advantages of
  that language;
\item has insufficient information about the cost associated with
  hiring new staff for a new language vs the productivity advantages of
  that language.
\end{itemize}

\end{frame}
%-----------------------------------------------------------
\begin{frame}\frametitle{Language vs implementation}

\end{frame}
%-----------------------------------------------------------
\begin{frame}\frametitle{Language vs implementation}

Language: A description of the syntax and semantics of conforming
programs, and of consequences of using non-conforming constructs.

Example of the latter: In C, obtaining a pointer outside of an array
has undefined consequences.  (It is interesting to contemplate why.)

\end{frame}
%-----------------------------------------------------------
\begin{frame}\frametitle{Language vs implementation}

Implementation: Software that accepts conforming programs and executes
them according to the language semantics, and that reports
non-conforming constructs where required by the language definition.

\end{frame}
%-----------------------------------------------------------

\begin{frame}
\frametitle{Thank you}

Questions?
\end{frame}

\end{document}
