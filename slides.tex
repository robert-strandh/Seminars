\documentclass{slides}
\usepackage[utf8]{inputenc}
\usepackage{graphics}
\usepackage{portland}
\usepackage{epsfig}
\usepackage{alltt}
\usepackage{moreverb}
\usepackage{url}
\usepackage[dvips,usenames]{color}

\definecolor{MyLightMagenta}{rgb}{1,0.7,1}
\definecolor{darkgreen}{rgb}{0.1,0.7,0.1}

\newcommand{\darkgreen}[1]{\textcolor{darkgreen}{#1}}
\newcommand{\red}[1]{\textcolor{red}{#1}}
\newcommand{\thistle}[1]{\textcolor{Thistle}{#1}}
\newcommand{\apricot}[1]{\textcolor{Apricot}{#1}}
\newcommand{\melon}[1]{\textcolor{Melon}{#1}}
\newcommand{\dandelion}[1]{\textcolor{Dandelion}{#1}}
\newcommand{\green}[1]{\textcolor{OliveGreen}{#1}}
\newcommand{\lavender}[1]{\textcolor{Lavender}{#1}}
\newcommand{\mylightmagenta}[1]{\textcolor{MyLightMagenta}{#1}}
\newcommand{\blue}[1]{\textcolor{RoyalBlue}{#1}}
\newcommand{\darkorchid}[1]{\textcolor{DarkOrchid}{#1}}
\newcommand{\orchid}[1]{\textcolor{Orchid}{#1}}
\newcommand{\brickred}[1]{\textcolor{BrickRed}{#1}}
\newcommand{\peach}[1]{\textcolor{Peach}{#1}}
\newcommand{\bittersweet}[1]{\textcolor{Bittersweet}{#1}}
\newcommand{\salmon}[1]{\textcolor{Salmon}{#1}}
\newcommand{\yelloworange}[1]{\textcolor{YellowOrange}{#1}}
\newcommand{\periwinkle}[1]{\textcolor{Periwinkle}{#1}}

\newcommand{\names}[1]{\periwinkle{#1}}
\newcommand{\motcle}[1]{\mylightmagenta{#1}}
\newcommand{\classname}[1]{\darkgreen{#1}}
\newcommand{\str}[1]{\yelloworange{#1}}
\newcommand{\defun}[1]{\orchid{#1}}
\newcommand{\ti}[1]{\begin{center}\Large{\textcolor{blue}{#1}}\end{center}}
\newcommand{\alert}[1]{\thistle{#1}}
\newcommand{\lispprint}[1]{\dandelion{#1}}
\newcommand{\lispvalue}[1]{\red{#1}}
\newcommand{\tr}[1]{\texttt{\red{#1}}}
\newcommand{\emc}[1]{\red{#1}}
\newcommand{\lispobj}[1]{\green{\texttt{#1}}}
\def\prompt{{\textcolor{Orchid}{CL-USER>}}}
\newcommand{\promptp}[1]{\textcolor{Orchid}{#1>}}

\newcommand{\Comment}[1]{
\begin{center}
\textcolor{yellow}
{#1}
\end{center}
}

\def\bs{$\backslash$}
\def\inputfig#1{\input #1}
\def\inputtex#1{\input #1}

\begin{document}
\landscape
\setlength{\oddsidemargin}{1cm}
\setlength{\evensidemargin}{1cm}
\setlength{\marginparwidth}{1cm}
\setlength{\parskip}{0.5cm}
\setlength{\parindent}{0cm}
%-----------------------------------------------------------
\begin{slide}\ti{Kaizen for everyone}
\vskip 0.5cm
\begin{center}
Robert Strandh \\
Université de Bordeaux \\
Bordeaux, France
\end{center}
\vfill\end{slide}
%-----------------------------------------------------------
\begin{slide}\ti{What is Kaizen?}

  \begin{itemize}
  \item A technique for continual improvement.
  \item Typically used in an industrial setting.
  \item This talk focuses on improving personal productivity.
  \end{itemize}

\vfill\end{slide}
%-----------------------------------------------------------
\begin{slide}\ti{Overview of talk}

Techniques applicable mainly to software developers and sophisticated
computer users.

  \begin{itemize}
  \item Very simple techniques based on customizing existing tools.
  \item More difficult techniques requiring some programming.
  \item Techniques that require organizational change.
  \end{itemize}

\vfill\end{slide}
%-----------------------------------------------------------
\begin{slide}\ti{Overview of talk}

The techniques discussed are \emph{examples} based on:

\begin{itemize}
\item work situation,
\item knowledge,
\item experience, and
\item preferred tools.
\end{itemize}

Each person must find his/her own applicable improvements.

\vfill\end{slide}
%-----------------------------------------------------------
\begin{slide}\ti{Simple techniques: Customize existing tools}

\vfill\end{slide}
%-----------------------------------------------------------
\begin{slide}\ti{Example (shell alias)}

How long does it take to type:

\texttt{git push github master}

followed by RETURN/ENTER?

\vfill\end{slide}
%-----------------------------------------------------------
\begin{slide}\ti{Example (shell alias)}

Compare to how long it takes to type:

\texttt{gip}

followed by RETURN/ENTER?

\vfill\end{slide}
%-----------------------------------------------------------
\begin{slide}\ti{Example (shell alias)}

At average typing speed, the first line takes around 9 seconds, and
the second one takes less than 2 seconds.

For simplicity, let's say we save 6 seconds.

Suppose further that we type this line 10 times per day.

This gives a total of 1 minute saved per day.

\vfill\end{slide}
%-----------------------------------------------------------
\begin{slide}\ti{Example (shell alias)}

1 minute per day is around 180 minutes per year, which is 3 hours.

\vfill\end{slide}
%-----------------------------------------------------------
\begin{slide}\ti{Is it worth the effort?}

Now imagine you find one such similar improvement to make every week.

After a month, your annual productivity gain is 12 hours, which is a
day and a half.

After a year, your annual productivity gain is 120 hours, which is 3
weeks.

\vfill\end{slide}
%-----------------------------------------------------------
\begin{slide}\ti{Is it worth the effort?}

How much time can you spend to make a single improvement worth 1
minute per day?

Assuming you want it to pay off within a year, you can spend 3 hours.

Think about these numbers for a while!!!!!

\vfill\end{slide}
%-----------------------------------------------------------
\begin{slide}\ti{Obstacles}

  \begin{itemize}
  \item Employer (may not approve)
  \item No incentive (paid by the hour)
  \item Psychological barriers
  \end{itemize}

\vfill\end{slide}
%-----------------------------------------------------------
\begin{slide}\ti{Example (editor abbrevs)}

Demo

\vfill\end{slide}
%-----------------------------------------------------------
\begin{slide}\ti{Estimate of time gain for French email}

I went through a month of archived outgoing email.

The saving was several hours per month, say 20 hours per year, which
is half a workweek.

During the years I was the head of the teaching program, I saved
roughly 4 workweeks.

\vfill\end{slide}
%-----------------------------------------------------------
\begin{slide}\ti{Example (abbrevs as for correcting spelling)}

\vfill\end{slide}
%-----------------------------------------------------------
\begin{slide}\ti{Learn to touch type}

Average typing speed is around 150 keystrokes per minute.

Consider the possibility of improving your typing speed by 10\%. 

If you spend 1 hour per day typing, you gain 6 minutes per day.

The result is half a workweek per year saved.

Consider that it takes a few weeks to get up to speed.

You learn faster if you move your keys around arbitrarily, or (like
me) if the markings on your keys do not correspond to the keyboard
layout.

\vfill\end{slide}
%-----------------------------------------------------------
\begin{slide}\ti{Adapt your keyboard}

As an Emacs user, I use the control key a lot.  So I put it where the
caps-lock key is normally located.  

Instead of using the meta key, I use C-[.

Instead of TAB, use C-i.

\vfill\end{slide}
%-----------------------------------------------------------
\begin{slide}\ti{Switch to a better editor}

Your editor should have at least the following features:

\begin{itemize}
\item Abbrevs
\item Dynamic abbrevs (M-/ in Emacs)
\item Spell checker
\item Completion
\item Customization for keyboard shortcuts
\item Programmable extensions
\end{itemize}


\vfill\end{slide}
%-----------------------------------------------------------
\begin{slide}\ti{More involved: Write some code}

Demo: Learning a new (natural) language.

\vfill\end{slide}
%-----------------------------------------------------------
\begin{slide}\ti{Estimate of time gain for learning Vietnamese}

The initial version of this application took around 2 hours to write.

During a couple of years, it easily saved me 100 dictionary lookups
per week.  

Estimate that it takes 30 seconds to look up an entry in an electronic
dictionary.

That comes to 50 minutes per week, let's say 1 hour.  The total gain
is a workweek per year.

\vfill\end{slide}
%-----------------------------------------------------------
\begin{slide}\ti{Other examples of writing some code}

  \begin{itemize}
  \item Balancing checking account.
  \end{itemize}

\vfill\end{slide}
%-----------------------------------------------------------
\begin{slide}\ti{Learn a new programming language}

Article by Paul Hudak and Mark Jones.

``Haskell vs. Ada vs. C++ vs. Awk vs. ... An Experiment in Software
Prototyping Productivity''

It is somewhat old, but the main idea is still valid.

\vfill\end{slide}
%-----------------------------------------------------------
\begin{slide}\ti{}

\vfill\end{slide}
%-----------------------------------------------------------
\begin{slide}\ti{Conclusions}

All the examples I have presented are conservative.  From observing
other people work, I can see that I am twice as fast as most of them,
just by knowing my editor and by typing faster.

To that should be added all the more complicated techniques that I
have invested in.

\vfill\end{slide}
%-----------------------------------------------------------
\begin{slide}\ti{Future work}

\vfill\end{slide}
%-----------------------------------------------------------
\begin{slide}\ti{Tomorrow's talk}

My ideal operating system.

\vfill\end{slide}
%-----------------------------------------------------------
\begin{slide}\ti{Possible future talks}

  \begin{itemize}
  \item Algorithms and data structures
  \item Object-oriented programming with generic functions
  \item Parsing technology
  \item Compiler technology (optimization)
  \item Operating-system techniques
  \item Automatic memory-management (garbage collection) techniques
  \item Non-automatic memory-management techniques
  \item Computer architecture
  \end{itemize}

\vfill\end{slide}
%% %-----------------------------------------------------------
%% \begin{slide}\ti{}

%% \vfill\end{slide}
%% %-----------------------------------------------------------
%% \begin{slide}\ti{}

%% \vfill\end{slide}
%% %-----------------------------------------------------------
%% \begin{slide}\ti{}

%% \vfill\end{slide}
%--------------------------------

\end{document}
 
